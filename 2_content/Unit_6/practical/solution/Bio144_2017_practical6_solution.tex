\documentclass[a4paper,12pt]{scrartcl}

\usepackage{amsmath,natbib}
\usepackage[utf8]{inputenc}
\usepackage{bm}
\usepackage{url}
\usepackage[printonlyused]{acronym}
\usepackage{Sweave}
\usepackage{color}
\usepackage{mathbbol}
\usepackage{numprint}
\usepackage{enumerate}
\usepackage{setspace}
\usepackage[left=2.5cm,top=2cm,right=3cm,bottom=3cm,bindingoffset=0.5cm]{geometry}
\usepackage{titlesec}
\usepackage{lineno}
%\usepackage{enumitem}

\usepackage{tabularx}


\def\logit{\mathsf{logit}}
\DeclareMathOperator{\E}{\mathsf{E}} 
\def\SBP{\text{SBP}}
\def\G{\text{G}}
\def\T{\text{T}}
\def\o{_{\text{obs}}}
\def\given{\,|\,}
\def\na{\tt{NA}}
\def\nin{\noindent}
\def\mm#1{\ensuremath{\boldsymbol{#1}}} % version: amsmath
\definecolor{grey}{rgb}{.6,.6,.6}
\newcommand{\p}{\operatorname{{p}}} % Density function
\newcommand{\Var}{\text{Var}} % Varianz
\newcommand{\Cor}{\text{Cor}} % correlation
\newcommand{\latin}[1]{\textit{#1}}
\newcommand{\abk}[1]{\mbox{#1}\xdot}
\DeclareRobustCommand\xdot{\futurelet\token\Xdot}
\def\Xdot{%
  \ifx\token\bgroup.%
  \else\ifx\token\egroup.%
  \else\ifx\token\/.%
  \else\ifx\token\ .%
  \else\ifx\token!.%
  \else\ifx\token,.%
  \else\ifx\token:.%
  \else\ifx\token;.%
  \else\ifx\token?.%
  \else\ifx\token/.%
  \else\ifx\token'.%
  \else\ifx\token).%
  \else\ifx\token-.%
  \else\ifx\token+.%
  \else\ifx\token~.%
  \else\ifx\token.%
  \else.\ %
  \fi\fi\fi\fi\fi\fi\fi\fi\fi\fi\fi\fi\fi\fi\fi\fi%
}
\newcommand{\eg}{\abk{\latin{e.\,g}}}
\newcommand{\ie}{\abk{\latin{i.\,e}}}
%\newcommand{\hl}{\textcolor{blue}} 
\newcommand{\hll}{\textcolor{blue}}

\titleformat*{\section}{\large\bfseries\sffamily}
\titleformat*{\subsection}{\large\bfseries\sffamily}
\titleformat*{\subsubsection}{\bfseries\sffamily}
\titleformat*{\paragraph}{\bfseries\sffamily}



\begin{document}
\begin{center}
{\bf Bio144, 6./7. April 2017} \\[8mm] 
{\large Solution to practical part 7: Exercise on linear algebra}\\[10mm]
\end{center}

 

\noindent{\bf 1.}
The matrices and vectors can be generated in R:\\

\begin{Schunk}
\begin{Sinput}
> A <- matrix(c(4,2,3,0,3,6),byrow=TRUE,nrow=2)
> B <- matrix(c(0,3,6,-1,-1,0),byrow=TRUE,nrow=2)
> x <- c(-1,2,-3)
> y <- c(5,3,-2)
\end{Sinput}
\end{Schunk}

\begin{enumerate}[a)]
\item 
\begin{Schunk}
\begin{Sinput}
> 2*A
\end{Sinput}
\begin{Soutput}
     [,1] [,2] [,3]
[1,]    8    4    6
[2,]    0    6   12
\end{Soutput}
\end{Schunk}
\item 
\begin{Schunk}
\begin{Sinput}
> A + B
\end{Sinput}
\begin{Soutput}
     [,1] [,2] [,3]
[1,]    4    5    9
[2,]   -1    2    6
\end{Soutput}
\end{Schunk}
\item
\begin{Schunk}
\begin{Sinput}
> A%*%t(B)
\end{Sinput}
\begin{Soutput}
     [,1] [,2]
[1,]   24   -6
[2,]   45   -3
\end{Soutput}
\end{Schunk}
\end{enumerate}



\end{document}
