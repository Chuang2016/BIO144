\documentclass[a4paper,12pt]{scrartcl}

\usepackage{amsmath,natbib}
\usepackage[utf8]{inputenc}
\usepackage{bm}
\usepackage{url}
\usepackage[printonlyused]{acronym}
\usepackage{Sweave}
\usepackage{color}
\usepackage{mathbbol}
\usepackage{numprint}
\usepackage{enumerate}
\usepackage{setspace}
\usepackage[left=2.5cm,top=2cm,right=3cm,bottom=3cm,bindingoffset=0.5cm]{geometry}
\usepackage{titlesec}
\usepackage{lineno}
%\usepackage{enumitem}

\usepackage{tabularx}


\def\logit{\mathsf{logit}}
\DeclareMathOperator{\E}{\mathsf{E}} 
\def\SBP{\text{SBP}}
\def\G{\text{G}}
\def\T{\text{T}}
\def\o{_{\text{obs}}}
\def\given{\,|\,}
\def\na{\tt{NA}}
\def\nin{\noindent}
\def\mm#1{\ensuremath{\boldsymbol{#1}}} % version: amsmath
\definecolor{grey}{rgb}{.6,.6,.6}
\newcommand{\p}{\operatorname{{p}}} % Density function
\newcommand{\Var}{\text{Var}} % Varianz
\newcommand{\Cor}{\text{Cor}} % correlation
\newcommand{\latin}[1]{\textit{#1}}
\newcommand{\abk}[1]{\mbox{#1}\xdot}
\DeclareRobustCommand\xdot{\futurelet\token\Xdot}
\def\Xdot{%
  \ifx\token\bgroup.%
  \else\ifx\token\egroup.%
  \else\ifx\token\/.%
  \else\ifx\token\ .%
  \else\ifx\token!.%
  \else\ifx\token,.%
  \else\ifx\token:.%
  \else\ifx\token;.%
  \else\ifx\token?.%
  \else\ifx\token/.%
  \else\ifx\token'.%
  \else\ifx\token).%
  \else\ifx\token-.%
  \else\ifx\token+.%
  \else\ifx\token~.%
  \else\ifx\token.%
  \else.\ %
  \fi\fi\fi\fi\fi\fi\fi\fi\fi\fi\fi\fi\fi\fi\fi\fi%
}
\newcommand{\eg}{\abk{\latin{e.\,g}}}
\newcommand{\ie}{\abk{\latin{i.\,e}}}
%\newcommand{\hl}{\textcolor{blue}} 
\newcommand{\hll}{\textcolor{blue}}

\titleformat*{\section}{\large\bfseries\sffamily}
\titleformat*{\subsection}{\large\bfseries\sffamily}
\titleformat*{\subsubsection}{\bfseries\sffamily}
\titleformat*{\paragraph}{\bfseries\sffamily}



\begin{document}
\begin{center}
{\bf Bio144, 6./7. April 2017} \\[8mm] 
{\large Solution to practical part 7: Exercise on linear algebra}\\[10mm]
\end{center}

 

\noindent{\bf 1.}
The matrices and vectors can be generated in R:

\begin{Schunk}
\begin{Sinput}
> A <- matrix(c(4,2,3,1,4,6),byrow=TRUE,nrow=2)
> B <- matrix(c(0,3,6,-1,-1,0),byrow=TRUE,nrow=2)
> x <- c(-1,2,-3)
> y <- c(5,3,-2)
\end{Sinput}
\end{Schunk}

\begin{enumerate}[a)]
\item 
\begin{Schunk}
\begin{Sinput}
> 2*A
\end{Sinput}
\begin{Soutput}
     [,1] [,2] [,3]
[1,]    8    4    6
[2,]    2    8   12
\end{Soutput}
\end{Schunk}
\item 
\begin{Schunk}
\begin{Sinput}
> A + B
\end{Sinput}
\begin{Soutput}
     [,1] [,2] [,3]
[1,]    4    5    9
[2,]    0    3    6
\end{Soutput}
\end{Schunk}
\item Calculating the results by hand: 
\begin{equation*}
\left[\begin{array}{ccc} 4 & 2 & 3 \\ 1 &4 & 6 \end{array}\right] \cdot \begin{bmatrix} 0 & -1 \\ 3 & -1 \\ 6 & 0\end{bmatrix}
= \begin{bmatrix}
4\cdot 0 + 2\cdot 3 + 3\cdot 6 \quad & \quad 4\cdot (-1) + 2\cdot (-1) + 3\cdot 0 \\
1\cdot 0 + 4\cdot 3 + 6\cdot 6 \quad & \quad 1\cdot (-1) + 4\cdot (-1) + 6\cdot 0 \\
\end{bmatrix} =  \begin{bmatrix}
24 &  -6 \\
48 & -5\\
\end{bmatrix} 
\end{equation*}
Checking with R:
\begin{Schunk}
\begin{Sinput}
> A%*%t(B)
\end{Sinput}
\begin{Soutput}
     [,1] [,2]
[1,]   24   -6
[2,]   48   -5
\end{Soutput}
\end{Schunk}
\item Again, start by hand:

\begin{equation*}
\left[\begin{array}{ccc} 4 & 2 & 3 \\ 1 &4 & 6 \end{array}\right] \cdot \begin{bmatrix}
-1 \\
2 \\
-3\\
\end{bmatrix}
=  
\begin{bmatrix}
4\cdot (-1) + 2\cdot 2 + 3\cdot (-3)\\
1\cdot (-1) + 4 \cdot 2 + 6\cdot (-3)
\end{bmatrix}
= 
\begin{bmatrix}
-9 \\ -11
\end{bmatrix}
\end{equation*}

And check by R:
\begin{Schunk}
\begin{Sinput}
> A %*% x
\end{Sinput}
\begin{Soutput}
     [,1]
[1,]   -9
[2,]  -11
\end{Soutput}
\end{Schunk}
\item Not defined, wrong dimensions.
\item Not defined, wrong dimensions.
\item 
\begin{Schunk}
\begin{Sinput}
> A %*% t(A)
\end{Sinput}
\begin{Soutput}
     [,1] [,2]
[1,]   29   30
[2,]   30   53
\end{Soutput}
\end{Schunk}
\item 
\begin{Schunk}
\begin{Sinput}
> t(A) %*% A
\end{Sinput}
\begin{Soutput}
     [,1] [,2] [,3]
[1,]   17   12   18
[2,]   12   20   30
[3,]   18   30   45
\end{Soutput}
\end{Schunk}
\item
\begin{Schunk}
\begin{Sinput}
> t(x)%*%x
\end{Sinput}
\begin{Soutput}
     [,1]
[1,]   14
\end{Soutput}
\end{Schunk}
\item
\begin{Schunk}
\begin{Sinput}
> x%*%t(x)
\end{Sinput}
\begin{Soutput}
     [,1] [,2] [,3]
[1,]    1   -2    3
[2,]   -2    4   -6
[3,]    3   -6    9
\end{Soutput}
\end{Schunk}
\end{enumerate}



\noindent{\bf 2.}
\begin{enumerate}[a)]
\item In the lecture we have defined the covariate vectors $\bm{x^{(1)}}$ and $\bm{x^{(2)}}$, the data matrix $\bm{\tilde{X}}$, the $\bm\beta$ vector, and the response vector:
\begin{Schunk}
\begin{Sinput}
> x1 <- c(0,1,2,3,4)
> x2 <- c(4,1,0,1,4)
> Xtilde <- matrix(c(rep(1,5),x1,x2),ncol=3)
> t.beta <- c(10,5,-2)
> t.y <- Xtilde%*%t.beta
> t.e <- rnorm(5,0,1)
> t.Y <- t.y  + t.e
> r.lm <- lm(t.Y ~ x1 + x2)
> summary(r.lm)$coef
\end{Sinput}
\begin{Soutput}
             Estimate Std. Error   t value    Pr(>|t|)
(Intercept) 10.424761  1.1593521  8.991885 0.012143152
x1           4.996217  0.3895548 12.825454 0.006024432
x2          -2.494555  0.3292339 -7.576848 0.016976675
\end{Soutput}
\begin{Sinput}
> solve(t(Xtilde) %*% Xtilde)  %*%  t(Xtilde) %*% t.Y
\end{Sinput}
\begin{Soutput}
          [,1]
[1,] 10.424761
[2,]  4.996217
[3,] -2.494555
\end{Soutput}
\end{Schunk}

\item Generate a matrix of true value \texttt{t.y} and a matrix of residual errors \texttt{t.E} (instead of generating a separate \texttt{t.e} each time). The 100 observed vectors are then stored in a $5\times 100$ matrix \texttt{t.Y}: 
\begin{Schunk}
\begin{Sinput}
> t.y <- matrix(rep(t.y,100),nrow=5,byrow=F)
> t.E <- matrix(rnorm(500,0,1),nrow=5)
> t.Y <- t.y  + t.E
\end{Sinput}
\end{Schunk}

% The \texttt{apply} functions applies the defined function on all columns of \texttt{t.Y} (note that if the second argument was \texttt{=1} the function would be applied to all rows) 
% <<>>=
% r.coef <- t(apply(t.Y, 2, FUN = function(y) lm(y ~ x1 + x2)$coefficients))
% @
100 iterations of the regression:
\begin{Schunk}
\begin{Sinput}
> r.coef <- matrix(NA,ncol=3, nrow=100)
> for (i in 1:100) {
+   r.coef[i,] <- lm(t.Y[,i] ~x1 + x2)$coefficients
+ }
\end{Sinput}
\end{Schunk}

\item We procude the required graphs using \texttt{ggplot}. Load the libraries:
\begin{Schunk}
\begin{Sinput}
> library(ggplot2)
> library(tidyr)
> library(dplyr)
\end{Sinput}
\end{Schunk}

For the histograms, first convert \texttt{r.coef} into a data frame and rename the columns:
\begin{Schunk}
\begin{Sinput}
> r.coef <- data.frame(r.coef)
> names(r.coef) <- c("beta0","beta1","beta2")
\end{Sinput}
\end{Schunk}
Then either produce three separate plots using
\begin{Schunk}
\begin{Sinput}
> ggplot(r.coef,aes(x=beta0)) + geom_histogram()
> ggplot(r.coef,aes(x=beta1)) + geom_histogram()
> ggplot(r.coef,aes(x=beta2)) + geom_histogram()
\end{Sinput}
\end{Schunk}

Or use the \texttt{gather()} function from the \texttt{tidyr} package, which has the advantage that you can then use \texttt{facet\_wrap()}:
\begin{Schunk}
\begin{Sinput}
> ggplot(gather(r.coef, key=variable, value=value), aes(value)) +
+   geom_histogram(bins=10) + facet_wrap(~variable, scales = "free") + 
+   theme_bw()
\end{Sinput}
\end{Schunk}
\includegraphics{Bio144_2017_practical6_solution-016}




% Plotting results using base R:
% \setkeys{Gin}{width=0.9\textwidth}
% <<fig=T,echo=T,width=6.5,height=3>>=
% par(mfrow=c(1,3))
% hist(r.coef[,1],xlab="beta0",main="")
% hist(r.coef[,2],xlab="beta1",main="")
% hist(r.coef[,3],xlab="beta2",main="")
% @
% 
% \eject
For the scatterplots we can continue to use \texttt{r.coef}:

\setkeys{Gin}{width=0.3\textwidth}
\begin{Schunk}
\begin{Sinput}
> ggplot(r.coef,aes(x=beta0,y=beta1)) + geom_point()  
\end{Sinput}
\end{Schunk}
\includegraphics{Bio144_2017_practical6_solution-017}
\begin{Schunk}
\begin{Sinput}
> ggplot(r.coef,aes(x=beta0,y=beta2)) + geom_point()
\end{Sinput}
\end{Schunk}
\includegraphics{Bio144_2017_practical6_solution-018}
\begin{Schunk}
\begin{Sinput}
> ggplot(r.coef,aes(x=beta1,y=beta2)) + geom_point()
\end{Sinput}
\end{Schunk}
\includegraphics{Bio144_2017_practical6_solution-019}

Observation:  $\beta_0$ seems to be correlated with $\beta_1$ and $\beta_2$.

\end{enumerate}

\end{document}
