\documentclass[a4paper,12pt]{scrartcl}

\usepackage{amsmath,natbib}
\usepackage[utf8]{inputenc}
\usepackage{bm}
\usepackage{url}
\usepackage[printonlyused]{acronym}
\usepackage{Sweave}
\usepackage{color}
\usepackage{mathbbol}
\usepackage{numprint}
\usepackage{enumerate}
\usepackage{setspace}
\usepackage[left=2.5cm,top=2cm,right=3cm,bottom=3cm,bindingoffset=0.5cm]{geometry}
\usepackage{titlesec}
\usepackage{lineno}
%\usepackage{enumitem}

\usepackage{tabularx}


\def\logit{\mathsf{logit}}
\DeclareMathOperator{\E}{\mathsf{E}} 
\def\SBP{\text{SBP}}
\def\G{\text{G}}
\def\T{\text{T}}
\def\o{_{\text{obs}}}
\def\given{\,|\,}
\def\na{\tt{NA}}
\def\nin{\noindent}
\def\mm#1{\ensuremath{\boldsymbol{#1}}} % version: amsmath
\definecolor{grey}{rgb}{.6,.6,.6}
\newcommand{\p}{\operatorname{{p}}} % Density function
\newcommand{\Var}{\text{Var}} % Varianz
\newcommand{\Cor}{\text{Cor}} % correlation
\newcommand{\latin}[1]{\textit{#1}}
\newcommand{\abk}[1]{\mbox{#1}\xdot}
\DeclareRobustCommand\xdot{\futurelet\token\Xdot}
\def\Xdot{%
  \ifx\token\bgroup.%
  \else\ifx\token\egroup.%
  \else\ifx\token\/.%
  \else\ifx\token\ .%
  \else\ifx\token!.%
  \else\ifx\token,.%
  \else\ifx\token:.%
  \else\ifx\token;.%
  \else\ifx\token?.%
  \else\ifx\token/.%
  \else\ifx\token'.%
  \else\ifx\token).%
  \else\ifx\token-.%
  \else\ifx\token+.%
  \else\ifx\token~.%
  \else\ifx\token.%
  \else.\ %
  \fi\fi\fi\fi\fi\fi\fi\fi\fi\fi\fi\fi\fi\fi\fi\fi%
}
\newcommand{\eg}{\abk{\latin{e.\,g}}}
\newcommand{\ie}{\abk{\latin{i.\,e}}}
%\newcommand{\hl}{\textcolor{blue}} 
\newcommand{\hll}{\textcolor{blue}}

\titleformat*{\section}{\large\bfseries\sffamily}
\titleformat*{\subsection}{\large\bfseries\sffamily}
\titleformat*{\subsubsection}{\bfseries\sffamily}
\titleformat*{\paragraph}{\bfseries\sffamily}



\begin{document}
\begin{center}
{\bf Bio144, 6./7. April 2017} \\[8mm] 
{\large Practical part 7: Exercise on linear algebra}\\[10mm]
\end{center}

 

\begin{enumerate}[\bf{1.}]
\item Let us use the following matrices and vectors:
\begin{equation*}
\mathbf{A}=\left[\begin{array}{ccc} 3 & 2 & 4 \\ 1 &4 & 6 \end{array}\right]\ , \quad
\mathbf{B} = \begin{bmatrix} 0 & 2 & 4 \\ -1 & -1 & 0\end{bmatrix}\ , \quad \bm{x}=\begin{bmatrix}-1 \\ 2 \\ -3 \end{bmatrix} \ , \quad \bm{y}=\begin{bmatrix}5 \\ 3 \\ -2 \end{bmatrix}
\end{equation*}

Calculate all of the following expressions, if they are defined. Solve at least a)-d) by hand.\\[2mm]
 

\begin{tabularx}{\columnwidth}{XXXl}%
 {\bf a)} $2\cdot \mathbf{A}$  & {\bf b)} $\mathbf{A}+\mathbf{B}$  & {\bf c)} $\mathbf{A}\cdot\mathbf{B}^\T$ & {\bf d)} $\mathbf{A}\cdot\bm{x}$\\[2mm]
 {\bf e)} $\mathbf{A}\cdot \mathbf{B}$ & {\bf f)} $\mathbf{B}^\T \cdot \bm{y}$ & {\bf g)} $\mathbf{A}\cdot \mathbf{A}^\T$ & {\bf h)}  $\mathbf{A}^\T\cdot \mathbf{A}$ \\[2mm]
 {\bf i)} $\bm{x}^\T\cdot\bm{x}$ &  {\bf j)} $\bm{x}\cdot\bm{x}^\T$\\[8mm]
\end{tabularx}
{\bf R-hints:}
\begin{itemize}
\item \texttt{t(A)} corresponds to $\mathbf{A}^\T$
\item Make sure you understand the difference between $\mathbf{A}$\texttt{*}$\mathbf{B}$ and $\mathbf{A}$\texttt{\%*\%}$\mathbf{B}$
\end{itemize}
 

\end{enumerate}
 

\end{document}
